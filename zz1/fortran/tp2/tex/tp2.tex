\documentclass{report}

\usepackage[utf8]{inputenc}
\usepackage[T1]{fontenc}
\usepackage[francais]{babel}
\usepackage{amssymb}
\usepackage{algorithm}
\usepackage{algorithmic}

\title{Compte rendu du TP2 de Fortran}
\author{\textsc{Laurent} Valentin}

\begin{document}

\maketitle

\tableofcontents

\setcounter{chapter}{1}
\newpage
\section{Objectifs de TP}\label{objectifs}
On cherche à résoudre le système linéaire
\begin{displaymath}
Ax = b
\end{displaymath}
On sait résoudre ce système grâce à des méthodes directes comme Gauss ou la décomposition LU\.
Pourtant ces methodes sont inefficaces pour des matrices creuses ou lorsque la taille de la matrice
est grande.

Ce TP a pour objectif de découvrir les methodes itératives permettant de résoudre ces systèmes plus efficacement.
Il s'agira de programmer ces methodes en Fortran et de les comparer afin de résoudre un systeme quelconque.

Ces méthodes fonctionnent sur le même principe : on part d'un vecteur $x^0$ et on calcule $r^0 = Ax^0-b$ un vecteur
de $\mathbb(R)^n$ appelé résidu du système. On itère alors ce principe $n$ fois jusqu'à ce qu'un critère d'arrêt soit 
vérifié. On a alors la suite de vecteurs $\{x^k\}_{k \geq 0}$.

Des théorèmes nous montrent alors que ces méthodes convergent vers un vecteur $x$ qui est solution du système $Ax = b$.
Ici, le critère d'arrêt sera choisi par l'utilisateur lors de l'initialisation de \og eps\fg dans \textit{syslin.f90}.
On comparera cette valeur à
\begin{displaymath}
\frac{\| x^{k+1} - x^{k} \|}{\| x^{k+1} \|}
\end{displaymath}
On choisira dans ce TP la norme 2 pour effectuer le test d'arrêt.

\section{Algorithmes importants}\label{algo}
On a 3 methodes importantes qui seront programmées dans ce TP : la methode de Jacobi, la methode de Gauss-Seidel 
et la methode de relaxation. Chacune de ces methodes respectent le principe décrit dans \ref{objectifs}.

\subsubsection{La methode de Jacobi}
\begin{algorithm}
\caption{Methode de Jacobi}
\begin{algorithmic} 
\REQUIRE la matrice $A$ et le vecteur $b$
\ENSURE le vecteur $x$ tel que $Ax = b$
\STATE $x^k \leftarrow x^0$
\WHILE{$\frac{\| x^{k+1} - x^{k} \|}{\| x^{k+1} \|} \geq \epsilon $}
\FOR{$i = 1 \to n$}
\STATE $x^{k+1}_{i} \leftarrow \frac{1}{a_{i,i}} [b_i - \displaystyle\sum^{i-1}_{j=1}{a_{i,j}x^{k}_{j}} - \displaystyle\sum^{n}_{j=i+1}{a_{i,j}x^{k}_{j}}]$
\ENDFOR
\ENDWHILE
\end{algorithmic}
\end{algorithm}

\newpage
\subsubsection{La methode de Gauss-Seidel}
\begin{algorithm}
\caption{Methode de Gauss-Seidel}
\begin{algorithmic} 
\REQUIRE la matrice $A$ et le vecteur $b$
\ENSURE le vecteur $x$ tel que $A*x = b$
\STATE $x^k \leftarrow x^0$
\WHILE{$\frac{\| x^{k+1} - x^{k} \|}{\| x^{k+1} \|} \geq \epsilon $}
\FOR{$i = 1 \to n$}
\STATE $x^{k+1}_{i} \leftarrow \frac{1}{a_{i,i}} [b_i - \displaystyle\sum^{i-1}_{j=1}{a_{i,j}*x^{k+1}_{j}} - \displaystyle\sum^{n}_{j=i+1}{a_{i,j}x^{k}_{j}}]$
\ENDFOR
\ENDWHILE
\end{algorithmic}
\end{algorithm}

\subsubsection{La methode de Relaxation}
On introduit dans cette methode le coefficient $\omega$ de relaxation afin \og d'améliorer \fg la methode de Gauss-Seidel
\begin{algorithm}
\caption{Methode de Relaxation}
\begin{algorithmic} 
\REQUIRE la matrice $A$ et le vecteur $b$
\ENSURE le vecteur $x$ tel que $Ax = b$
\STATE $x^k \leftarrow x^0$
\WHILE{$\frac{\| x^{k+1} - x^{k} \|}{\| x^{k+1} \|} \geq \epsilon $}
\FOR{$i = 1 \to n$}
\STATE $x^{k+1}_{i} \leftarrow (1 - \omega)x^{k}_{i} + \frac{\omega}{a_{i,i}} [b_i - \displaystyle\sum^{i-1}_{j=1}{a_{i,j}x^{k+1}_{j}} - \displaystyle\sum^{n}_{j=i+1}{a_{i,j}x^{k}_{j}}]$
\ENDFOR
\ENDWHILE
\end{algorithmic}
\end{algorithm}


\newpage
\section{Listing en Fortran}
\subsection{Programme principal : syslin.f90}
Le programme affiche le menu, effectue la demande "ichoix" et appelle
les modules "methode.f90" et "affichage.f90".
\begin{small}
\begin{verbatim}
!------------------------------------------------------------------------------
! TP2 : Méthodes itératives de résolution des systèmes linéaires
! Auteur : Valentin Laurent (ZZ1,G1)
! Date : 11/12/12
! Compilation : gfortran methode.f90 affichage.f90 syslin.f90 -o syslin
! Fichiers de tests : matrice4.txt et matrice9.txt
!------------------------------------------------------------------------------

PROGRAM syslin

USE methodes
USE affichage

IMPLICIT NONE

integer                                 :: ichoix   !choix de l'utilisateur
real(8),dimension(:,:),allocatable      :: A        !matrice A du système
real(8),dimension(:),allocatable        :: b        !second membre du système
real(8),dimension(:),allocatable        :: x        !inconnue du système
real(8)                                 :: omega    !coefficient de relaxation
integer                                 :: nb_iter  !nombre d'itération
integer                                 :: n        !rang de A
integer                                 :: i,j,k    !variable de boucle
integer                                 :: menu     !valeur de menu
character (len=20)                      :: nomfich  !nom du fichier
integer                                 :: ierr     !variable d'erreur
real(8)                                 :: x0       !valeur de x0
real(8)                                 :: eps      !valeur de test


!initialisation du menu
menu = 1

! Lecture du nom du fichier de donnees
print *,' Entrez le nom du fichier de donnees : '
read *,nomfich

!initialisation de x
print *,' Entrez la valeur de x0 (0.d0) :'
read *,x0

!initialisation de eps
print *,' Entrez la valeur de epsilon (1.d-12) :'
read *,eps

DO WHILE (menu /= 0)

    ! Ouverture du fichier en lecture
    open(1,file=nomfich,status='old',action='read',iostat=ierr)
    IF (ierr /= 0) THEN
        print *,'Fichier de donnees inexistant'
        stop
    END IF

    !lecture du rang
    read(1,*) n

    IF (menu == 1) THEN
        !allocation de A,b,x pour la première utilisation
        allocate(A(n,n))
        allocate(b(n))
        allocate(x(n))
    END IF

    !Lecture de la matrice A et de b dans "matrice.txt"
    DO i=1,n
        read(1,*) (A(i,j),j=1,n)
    END DO
    read(1,*) (b(k),k=1,n)

    !fermeture du fichier
    close(1)

    !initialisation de x
    x(1:n) = x0

    !affichage du menu
    print *,"----------------- tp2 Resolution de systemes lineaires -----------------"
    print *,"ichoix = 0 -> quitte le programme"
    print *,"ichoix = 1 -> Methode de Jacobi"
    print *,"ichoix = 2 -> Methode de Gauss-Seidel"
    print *,"ichoix = 3 -> Methode de relaxation"
    print *,"Le choix 3 effectue un tableau pour omega variant de 0.1 à 1.9"
    print*,"-------------------------------------------------------------------------"

    !saisie utilisateur
    print *,"Veuillez saisir votre choix ?"
    read *,ichoix

    !choix des différentes methodes
    SELECT CASE(ichoix)
        CASE (0)
            !sortie du programme
            menu = -1

        CASE (1)
            !methode de Jacobi
            call resolution(1,A,b,omega,nb_iter,x,eps)

            !affichage
            call afficheMat(A,n)
            call afficheVect(x,n)
            print *,"Nombre d'iterations : ",nb_iter

        CASE(2)
            !methode de Gauss-Seidel
            omega = 1.d0
            call resolution(2,A,b,omega,nb_iter,x,eps)

            !affichage
            call afficheMat(A,n)
            call afficheVect(x,n)
            print *,"Nombre d'iterations : ",nb_iter

        CASE(3)
            !methode de relaxation
            omega = 1.d-1
            call afficheMat(A,n)

            DO k=1,19

                !initialisation de x
                x(1:n) = x0

                call resolution(3,A,b,omega,nb_iter,x,eps)

                !affichage
                print '(/)'
                print *,"Valeur d'omega : ",omega
                call afficheVect(x,n)
                print *,"Nombre d'iterations : ",nb_iter
                omega = omega + 1.d-1
            END DO

        CASE DEFAULT
            print *,"Evenement inconnu"

    END SELECT

    !affichage
    menu = menu + 1
    print '(/)'

END DO

print *,"Fin du programme"
print '(/)'
!fin du programme principal (execution)

END PROGRAM syslin
!fin du TP2
\end{verbatim}
\end{small}
\subsection{Module : methode.f90}
Ce module effectue les methodes de Jacobi, Gauss-Seidel et de relaxation à selon
la valeur de "ichoix".
\begin{small}
\begin{verbatim}
MODULE methodes

CONTAINS

!------------------------------------------------------------------------------
! subroutine des différentes methodes de résolution du système linéaire Ax=b
!------------------------------------------------------------------------------

SUBROUTINE resolution(ichoix,A,b,omega,nb_iter,x1,eps)
IMPLICIT NONE

integer,intent(in)                      :: ichoix           !choix de l'utilisateur
real(8),dimension(:,:),intent(inout)    :: A                !matrice A
real(8),dimension(:),intent(inout)      :: x1               !inconnue
real(8),dimension(:),intent(in)         :: b                !second membre
real(8),intent(inout)                   :: omega            !coefficient de relaxation
real(8),dimension(:),allocatable        :: x2               !vecteur de la suite

integer,intent(out)                     :: nb_iter          !nombre d'itération
integer                                 :: i,j,k,im,ip      !variable de boucle
integer                                 :: n                !rang de A
real(8),intent(in)                      :: eps              !valeur de test
real(8)                                 :: arret            !critère d'arrêt
real(8)                                 :: v1, v2           !calcul des normes

!initialisation
nb_iter = 0
v1 = 0.d0
v2 = 0.d0
n = size(b)
allocate(x2(n))

arret = eps+1.d0

!methode de Jacobi
IF (ichoix == 1) THEN
    DO WHILE (arret > eps) !on utilise la norme 2

        !initialisation des variables de boucles (calcul de norme et itérations)
        v1 = 0.d0
        v2 = 0.d0
        nb_iter = nb_iter + 1

        !calcul des xk suivant la methode de Jacobi
        DO i=1,n
            ip = i+1
            im = i-1
            x2(i) = (b(i) - dot_product(A(i,1:im),x1(1:im)) &
                          - dot_product(A(i,ip:n),x1(ip:n)))/A(i,i)

            !calcul de la norme 2
            v1 = v1 + (x2(i)-x1(i))**2
            v2 = v2 + x2(i)**2
        END DO

        !calcul du critère d'arrêt (actualisé pour chaque itération)
        arret = sqrt(v1/v2)
        x1 = x2
    END DO

!methode de relaxation + Gauss-Seidel (omega = 1)
ELSE IF ((ichoix == 3) .or. (ichoix == 2)) THEN
    DO WHILE (arret > eps)

        !initialisation des variables de boucles (calcul de norme et itérations)
        v1 = 0.d0
        v2 = 0.d0
        nb_iter = nb_iter + 1

        !calcul des xk suivant la methode de Gauss-Seidel
        DO i=1,n
            ip = i+1
            im = i-1
            x2(i) = (1.d0-omega)*x1(i) + (b(i) - dot_product(A(i,1:im),x2(1:im)) &
                                               - dot_product(A(i,ip:n),x1(ip:n)))*omega/A(i,i)

            !calcul de la norme 2
            v1 = v1 + (x2(i)-x1(i))**2
            v2 = v2 + x2(i)**2
        END DO

        !calcul du critère d'arrêt (actualisé pour chaque itération)
        arret = sqrt(v1/v2)
        x1 = x2
    END DO
END IF

END SUBROUTINE resolution

!------------------------------------------------------------------------------
!------------------------------------------------------------------------------

END MODULE methodes
\end{verbatim}
\end{small}

\subsection{Module d'affichage : affichage.f90}
Le programme d'affichage provient du TP1, j'ai crée un module que je peux réutiliser
dans chaque programme afin d'afficher un vecteur ou une matrice.
\begin{small}
\begin{verbatim}
!------------------------------------------------------------------------------
! module d'affichage des matrices et des vecteurs
!------------------------------------------------------------------------------

MODULE affichage

CONTAINS

!------------------------------------------------------------------------------
!subroutine du l'affichage de la matrice a
!------------------------------------------------------------------------------

SUBROUTINE afficheMat(a,n)
IMPLICIT NONE

!déclaration des variables
real(8),dimension(:,:),intent(inout)        :: a    !matrice
integer, intent(inout)                      :: n    !rang de a
integer                                     :: i,j  !indice de boucle

!affichage de la matrice a
print *,"La matrice A :"
do i=1,n
   print *,(a(i,j), j=1,n)
end do

END SUBROUTINE afficheMat

!------------------------------------------------------------------------------
!subroutine de l'affichage du vecteur x
!------------------------------------------------------------------------------

SUBROUTINE afficheVect(x,n)
IMPLICIT NONE

!déclaration des variables
real(8),dimension(:),intent(inout)          :: x    !vecteur
integer,intent(inout)                       :: n    !rang de x
integer                                     :: i    !indice de boucle

!affichage du vecteur x
print *,"Le vecteur x :"
do i=1,n
    print *,x(i)
end do

END SUBROUTINE afficheVect

!------------------------------------------------------------------------------
!------------------------------------------------------------------------------

END MODULE affichage
\end{verbatim}
\end{small}

\newpage
\section{Jeux d'essais}
\subsection{Matrice de taille 4*4 (matrice4.txt)}
On utilise la matrice exmple :

$A=\left[\begin{array}{rrrr}
 4 & -1 & -1 &  0 \\
-1 &  4 &  0 & -1 \\
-1 &  0 &  4 & -1 \\
 0 & -1 & -1 &  4
\end{array}\right]$
et $b=\left[\begin{array}{rrrr}
4 \\
4 \\
4 \\
4 
\end{array}\right]$

\subsubsection{Methode de Jacobi}
\begin{small}
\begin{verbatim}
  Entrez le nom du fichier de donnees : 
matrice4.txt
  Entrez la valeur de x0 (0.d0) :
0
  Entrez la valeur de epsilon (1.d-12) :
1.d-12
 ----------------- tp2 Resolution de systemes lineaires -----------------
 ichoix = 0 -> quitte le programme
 ichoix = 1 -> Methode de Jacobi
 ichoix = 2 -> Methode de Gauss-Seidel
 ichoix = 3 -> Methode de relaxation
 Le choix 3 effectue un tableau pour omega variant de 0.1 ? 1.9
 -------------------------------------------------------------------------
 Veuillez saisir votre choix ?
1
 Le vecteur x :
   1.9999999999981810
   1.9999999999981810
   1.9999999999981810
   1.9999999999981810
 Nombre d'iterations :           40
\end{verbatim}
\end{small}

\subsubsection{Methode de Gauss-Seidel}
\begin{small}
\begin{verbatim}
 Veuillez saisir votre choix ?
2
 Le vecteur x :
   1.9999999999996589
   1.9999999999998295
   1.9999999999998295
   1.9999999999999147
 Nombre d'iterations :           22
\end{verbatim}

\subsubsection{Methode de Relaxation}
\begin{verbatim}
 Veuillez saisir votre choix ?
3


 Valeur d'omega :   0.10000000000000001     
 Le vecteur x :
   1.9999999999634070     
   1.9999999999643583     
   1.9999999999643583     
   1.9999999999652842     
 Nombre d'iterations :          470


 Valeur d'omega :   0.20000000000000001     
 Le vecteur x :
   1.9999999999829465     
   1.9999999999836455     
   1.9999999999836455     
   1.9999999999843283     
 Nombre d'iterations :            7


 Valeur d'omega :   0.30000000000000004     
 Le vecteur x :
   1.9999999999897882     
   1.9999999999903864     
   1.9999999999903864     
   1.9999999999909703     
 Nombre d'iterations :            3


 Valeur d'omega :   0.40000000000000002     
 Le vecteur x :
   1.9999999999936748     
   1.9999999999941904     
   1.9999999999941904     
   1.9999999999946902     
 Nombre d'iterations :            2


 Valeur d'omega :   0.50000000000000000     
 Le vecteur x :
   1.9999999999953850     
   1.9999999999958546     
   1.9999999999958546     
   1.9999999999963087     
 Nombre d'iterations :            1


 Valeur d'omega :   0.59999999999999998     
 Le vecteur x :
   1.9999999999969105     
   1.9999999999973248     
   1.9999999999973248     
   1.9999999999977209     
 Nombre d'iterations :            1


 Valeur d'omega :   0.69999999999999996     
 Le vecteur x :
   1.9999999999981368     
   1.9999999999984723     
   1.9999999999984723     
   1.9999999999987816     
 Nombre d'iterations :            1


 Valeur d'omega :   0.79999999999999993     
 Le vecteur x :
   1.9999999999990163     
   1.9999999999992539     
   1.9999999999992539     
   1.9999999999994578     
 Nombre d'iterations :            1


 Valeur d'omega :   0.89999999999999991     
 Le vecteur x :
   1.9999999999995659     
   1.9999999999997056     
   1.9999999999997056     
   1.9999999999998133     
 Nombre d'iterations :            1


 Valeur d'omega :   0.99999999999999989     
 Le vecteur x :
   1.9999999999998528     
   1.9999999999999165     
   1.9999999999999165     
   1.9999999999999583     
 Nombre d'iterations :            1


 Valeur d'omega :    1.0999999999999999     
 Le vecteur x :
   1.9999999999999689     
   1.9999999999999885     
   1.9999999999999885     
   1.9999999999999980     
 Nombre d'iterations :            1


 Valeur d'omega :    1.2000000000000000     
 Le vecteur x :
   1.9999999999999991     
   2.0000000000000013     
   2.0000000000000013     
   2.0000000000000013     
 Nombre d'iterations :            1


 Valeur d'omega :    1.3000000000000000     
 Le vecteur x :
   2.0000000000000018     
   2.0000000000000009     
   2.0000000000000009     
   2.0000000000000000     
 Nombre d'iterations :            1


 Valeur d'omega :    1.4000000000000001     
 Le vecteur x :
   1.9999999999999998     
   1.9999999999999996     
   1.9999999999999996     
   1.9999999999999996     
 Nombre d'iterations :            1


 Valeur d'omega :    1.5000000000000002     
 Le vecteur x :
   1.9999999999999998     
   2.0000000000000000     
   2.0000000000000000     
   2.0000000000000000     
 Nombre d'iterations :            1


 Valeur d'omega :    1.6000000000000003     
 Le vecteur x :
   2.0000000000000000     
   2.0000000000000000     
   2.0000000000000000     
   2.0000000000000000     
 Nombre d'iterations :            1


 Valeur d'omega :    1.7000000000000004     
 Le vecteur x :
   2.0000000000000000     
   2.0000000000000000     
   2.0000000000000000     
   2.0000000000000000     
 Nombre d'iterations :            1


 Valeur d'omega :    1.8000000000000005     
 Le vecteur x :
   2.0000000000000000     
   2.0000000000000000     
   2.0000000000000000     
   2.0000000000000000     
 Nombre d'iterations :            1


 Valeur d'omega :    1.9000000000000006     
 Le vecteur x :
   2.0000000000000000
   2.0000000000000000
   2.0000000000000000
   2.0000000000000000
 Nombre d'iterations :            1
\end{verbatim}
\end{small}

\newpage
\subsection{Matrice de taille 9*9 (matrice9.txt)}
On utilise la matrice exmple :


$A=\left[\begin{array}{rrrrrrrrr}
 4 & -1 &  0 & -1 &  0 &  0 &  0 &  0 &  0 \\
-1 &  4 & -1 &  0 & -1 &  0 &  0 &  0 &  0 \\
 0 & -1 &  4 &  0 &  0 & -1 &  0 &  0 &  0 \\
-1 &  0 &  0 &  4 & -1 &  0 & -1 &  0 &  0 \\
 0 & -1 &  0 & -1 &  4 & -1 &  0 & -1 &  0 \\
 0 &  0 & -1 &  0 & -1 &  4 &  0 &  0 & -1 \\
 0 &  0 &  0 & -1 &  0 &  0 &  4 & -1 &  0 \\
 0 &  0 &  0 &  0 & -1 &  0 & -1 &  4 & -1 \\
 0 &  0 &  0 &  0 &  0 & -1 &  0 & -1 &  4
\end{array}\right]$
et $b=\left[\begin{array}{rrrrrrrrr}
4 \\
4 \\
4 \\
4 \\
4 \\
4 \\
4 \\
4 \\
4
\end{array}\right]$

\subsubsection{Methode de Jacobi}
\begin{small}
\begin{verbatim}
  Entrez le nom du fichier de donnees : 
matrice9.txt
  Entrez la valeur de x0 (0.d0) :
0
  Entrez la valeur de epsilon (1.d-12) :
1.d-12
 ----------------- tp2 Resolution de systemes lineaires -----------------
 ichoix = 0 -> quitte le programme
 ichoix = 1 -> Methode de Jacobi
 ichoix = 2 -> Methode de Gauss-Seidel
 ichoix = 3 -> Methode de relaxation
 Le choix 3 effectue un tableau pour omega variant de 0.1 ? 1.9
 -------------------------------------------------------------------------
 Veuillez saisir votre choix ?
1
 Le vecteur x :
   2.7499999999954525     
   3.4999999999936335     
   2.7499999999954525     
   3.4999999999936335     
   4.4999999999909051     
   3.4999999999936335     
   2.7499999999954525     
   3.4999999999936335     
   2.7499999999954525     
 Nombre d'iterations :           78
\end{verbatim}
\end{small}

\subsubsection{Methode de Gauss-Seidel}
\begin{small}
\begin{verbatim}
 Veuillez saisir votre choix ?
2
 Le vecteur x :
   2.7499999999971578     
   3.4999999999971578     
   2.7499999999985789     
   3.4999999999971578     
   4.4999999999971578     
   3.4999999999985789     
   2.7499999999985789     
   3.4999999999985789     
   2.7499999999992895     
 Nombre d'iterations :           41
\end{verbatim}
\end{small}

\subsubsection{Methode de relaxation}
\begin{small}
\begin{verbatim}
 Veuillez saisir votre choix ?
3


 Valeur d'omega :   0.10000000000000001     
 Le vecteur x :
   2.7499999999190963     
   3.4999999998873359     
   2.7499999999215534     
   3.4999999998873359     
   4.4999999998431068     
   3.4999999998907576     
   2.7499999999215534     
   3.4999999998907576     
   2.7499999999239364     
 Nombre d'iterations :          784


 Valeur d'omega :   0.20000000000000001     
 Le vecteur x :
   2.7499999999622147     
   3.4999999999479052     
   2.7499999999640998     
   3.4999999999479052     
   4.4999999999281988     
   3.4999999999505360     
   2.7499999999640998     
   3.4999999999505360     
   2.7499999999659348     
 Nombre d'iterations :           12


 Valeur d'omega :   0.30000000000000004     
 Le vecteur x :
   2.7499999999769607     
   3.4999999999685896     
   2.7499999999786078     
   3.4999999999685896     
   4.4999999999572156     
   3.4999999999708877     
   2.7499999999786078     
   3.4999999999708877     
   2.7499999999802096     
 Nombre d'iterations :            5


 Valeur d'omega :   0.40000000000000002     
 Le vecteur x :
   2.7499999999847571     
   3.4999999999794960     
   2.7499999999862332     
   3.4999999999794960     
   4.4999999999724674     
   3.4999999999815494     
   2.7499999999862332     
   3.4999999999815494     
   2.7499999999876605     
 Nombre d'iterations :            3


 Valeur d'omega :   0.50000000000000000     
 Le vecteur x :
   2.7499999999872524     
   3.4999999999829923     
   2.7499999999886846     
   3.4999999999829923     
   4.4999999999773692     
   3.4999999999849889     
   2.7499999999886846     
   3.4999999999849889     
   2.7499999999900773     
 Nombre d'iterations :            1


 Valeur d'omega :   0.59999999999999998     
 Le vecteur x :
   2.7499999999918918     
   3.4999999999894573     
   2.7499999999931850     
   3.4999999999894573     
   4.4999999999863691     
   3.4999999999912426     
   2.7499999999931850     
   3.4999999999912426     
   2.7499999999944111     
 Nombre d'iterations :            2


 Valeur d'omega :   0.69999999999999996     
 Le vecteur x :
   2.7499999999938778     
   3.4999999999921876     
   2.7499999999950555     
   3.4999999999921876     
   4.4999999999901110     
   3.4999999999937987     
   2.7499999999950555     
   3.4999999999937987     
   2.7499999999961529     
 Nombre d'iterations :            1


 Valeur d'omega :   0.79999999999999993     
 Le vecteur x :
   2.7499999999956506     
   3.4999999999946012     
   2.7499999999966911     
   3.4999999999946012     
   4.4999999999933822     
   3.4999999999960050     
   2.7499999999966911     
   3.4999999999960050     
   2.7499999999976326     
 Nombre d'iterations :            1


 Valeur d'omega :   0.89999999999999991     
 Le vecteur x :
   2.7499999999971356     
   3.4999999999965827     
   2.7499999999980016     
   3.4999999999965827     
   4.4999999999960023     
   3.4999999999977183     
   2.7499999999980016     
   3.4999999999977183     
   2.7499999999987366     
 Nombre d'iterations :            1


 Valeur d'omega :   0.99999999999999989     
 Le vecteur x :
   2.7499999999982911     
   3.4999999999980735     
   2.7499999999989480     
   3.4999999999980735     
   4.4999999999978959     
   3.4999999999988951     
   2.7499999999989480     
   3.4999999999988951     
   2.7499999999994476     
 Nombre d'iterations :            1


 Valeur d'omega :    1.0999999999999999     
 Le vecteur x :
   2.7499999999991109     
   3.4999999999990798     
   2.7499999999995479     
   3.4999999999990798     
   4.4999999999990967     
   3.4999999999995857     
   2.7499999999995479     
   3.4999999999995857     
   2.7499999999998277     
 Nombre d'iterations :            1


 Valeur d'omega :    1.2000000000000000     
 Le vecteur x :
   2.7499999999996256     
   3.4999999999996647     
   2.7499999999998659     
   3.4999999999996647     
   4.4999999999997309     
   3.4999999999999098     
   2.7499999999998659     
   3.4999999999999098     
   2.7499999999999805     
 Nombre d'iterations :            1


 Valeur d'omega :    1.3000000000000000     
 Le vecteur x :
   2.7499999999998939     
   3.4999999999999352     
   2.7499999999999902     
   3.4999999999999352     
   4.4999999999999805     
   3.5000000000000115     
   2.7499999999999902     
   3.5000000000000115     
   2.7500000000000133     
 Nombre d'iterations :            1


 Valeur d'omega :    1.4000000000000001     
 Le vecteur x :
   2.7499999999999969     
   3.5000000000000151     
   2.7500000000000133     
   3.5000000000000151     
   4.5000000000000266     
   3.5000000000000142     
   2.7500000000000133     
   3.5000000000000142     
   2.7500000000000044     
 Nombre d'iterations :            1


 Valeur d'omega :    1.5000000000000002     
 Le vecteur x :
   2.7500000000000124     
   3.5000000000000120     
   2.7500000000000036     
   3.5000000000000120     
   4.5000000000000071     
   3.4999999999999982     
   2.7500000000000036     
   3.4999999999999982     
   2.7499999999999964     
 Nombre d'iterations :            1


 Valeur d'omega :    1.6000000000000003     
 Le vecteur x :
   2.7500000000000027     
   3.4999999999999987     
   2.7499999999999964     
   3.4999999999999987     
   4.4999999999999929     
   3.4999999999999951     
   2.7499999999999964     
   3.4999999999999951     
   2.7499999999999982     
 Nombre d'iterations :            1


 Valeur d'omega :    1.7000000000000004     
 Le vecteur x :
   2.7499999999999969     
   3.4999999999999951     
   2.7499999999999973     
   3.4999999999999951     
   4.4999999999999956     
   3.4999999999999996     
   2.7499999999999973     
   3.4999999999999996     
   2.7500000000000009     
 Nombre d'iterations :            1


 Valeur d'omega :    1.8000000000000005     
 Le vecteur x :
   2.7499999999999978     
   3.5000000000000004     
   2.7500000000000018     
   3.5000000000000004     
   4.5000000000000027     
   3.5000000000000022     
   2.7500000000000018     
   3.5000000000000022     
   2.7500000000000009     
 Nombre d'iterations :            1


 Valeur d'omega :    1.9000000000000006     
 Le vecteur x :
   2.7500000000000018     
   3.5000000000000031     
   2.7500000000000009     
   3.5000000000000031     
   4.5000000000000027     
   3.5000000000000000     
   2.7500000000000009     
   3.5000000000000000     
   2.7499999999999991     
 Nombre d'iterations :            1
\end{verbatim}
\end{small}
\newpage
\section{Conclusions}
On observe une nette décroissance du nombre d'itération effectués selon les methodes utilisées.
En effet, la methode de Jacobi utilise plus d'itération que la methode de Gauss-Seidel qui est
elle même moins efficace que la methode de relaxation.
Pourtant, même si ces methodes gagnent en efficacité elles présentent quelques inconvénients : 
\begin{itemize}
\item La methode de Jacobi nécessite le stockage de $x^k$ et $x^{k+1}$, ce qui prend plus de place dans la mémoire.
\item La methode de Gauss-Seidel est encore trop séquentielle et perd en efficacité.
\item La methode de relaxation est optimisée pour une certaine valeur de $\omega$ qu'il faut trouver.
\end{itemize}

On remarque, de plus, que la methode de relaxation est plus optimisée pour une valeur de $\omega$ proche de $1.d0$.

Le programme Fortran facilite la manipulation des matrices et des formules mathématiques complexes (surtout au niveau
du séquntiel). Il pose néammoins des problèmes lors de l'affichage. Par exemple, l'affichage des matrices de taille 4
et de taille 9 impose 8 chiffres après la virgule, j'ai essayé de changer de format (F12.8) mais l'affichage se fait
alors par colonnes.

On peut comparer ces methodes avec la methode LU présentée dans le TP2 pour la matrice de taille 4 et celle de taille 9,
on remarque alors que le nombre d'itération est très réduit.
\end{document}
