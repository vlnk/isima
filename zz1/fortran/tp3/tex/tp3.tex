\documentclass{report}

\usepackage[utf8]{inputenc}
\usepackage[T1]{fontenc}
\usepackage[francais]{babel}
\usepackage{amssymb}
\usepackage{algorithm}
\usepackage{algorithmic}

\title{Compte rendu du TP3 de Fortran}
\author{\textsc{Laurent} Valentin}

\begin{document}

\maketitle

\tableofcontents

\setcounter{chapter}{1}
\newpage
\section{Objectifs de TP}\label{objectifs}
L'objectif du TP est de calculer les valeurs propres d'une matrice quelconque à l'aide de la methode QR.
On part de la matrice $A = A_{0}$ et on écrit sa factorisation $QR$ par la methode de Householder, on a alors
$A_{0} = Q_{0} R_{0}$ et on calcule $A_{1} = R_{0} Q_{0}$, on factorise alors $A_{1}$ et on itère cette methode.

Ainsi a l'itération $k$ on a $A_{k} = R_{k}Q{k} = Q_{k}^{t}A_{k-1}Q_{k}$

On sait (par théorème) que toute matrice admet une factorisation $QR$ selon la methode de Householder, de plus
(encore par théorème) on peut montrer que les éléments diagonnaux des matrices $A_{k}$ convergent vers les valeurs
propres de la matrices $A$ avec les éléments non diagonaux qui convergent vers $0$.

Cette methode, très pratique pour calculer les valeurs propres d'une matrice quelconque (jusqu'à l'ordre 25 pour un coût
avantageux) sera traduite en fortran dans ce TP.
\section{Algorithmes importants}\label{algo}

\begin{algorithm}
\caption{Methode QR}
\begin{algorithmic} 
\REQUIRE la matrice $A$
\ENSURE le spectre de $A$
\STATE $Q \leftarrow I $
\WHILE{$max\{a_{i,j}\}_{i \neq j} \geq \epsilon $}
\FOR{$k = 1 \to n-1$}
\STATE $s \leftarrow \sqrt{\displaystyle\sum^{n}_{j=k}{a_{j,k}^2}}$
\STATE $r \leftarrow \sqrt{2s(s + \| a_{k,k} \|)}$
\STATE $w^{k} \leftarrow v^{k}/{r}$
\STATE $H_{i,j} \leftarrow \delta_{i,j} - 2 \omega_{i}^{k} \omega_{j}^{k}$
\STATE $Q \leftarrow Q H$
\STATE $A \leftarrow H A$
\ENDFOR
\STATE $A \leftarrow A Q$
\ENDWHILE
\end{algorithmic}
\end{algorithm}

\newpage
\section{Listing en Fortran}\label{fortran}
\subsection{Programme principal : val\_p.f90}
Le programme affiche le menu, effectue directement le calcul des valeurs propres avec la matrice QR à partir de
la matrice saisie par l'utilisateur. Ce programme appelle les modules affichage.f90 et algo.f90.

\begin{small}
\begin{verbatim}
!------------------------------------------------------------------------------
! TP3 : LES Méthodes itératives QR
! Auteur : Valentin Laurent (ZZ1,G1)
! Date : 11/12/12
! Compilation : gfortran algo.f90 affichage.f90 val_p.f90 -o val_p
! Fichiers de tests : matrice41.txt, matrice42.txt et matrice9.txt
!------------------------------------------------------------------------------

PROGRAM val_p

USE algo
USE affichage

IMPLICIT NONE

real(8),dimension(:,:),allocatable      :: A        !la matrice A
real(8),dimension(:), allocatable       :: spA      !le spectre de A
integer                                 :: n        !le rang de A
integer                                 :: i        !variable de boucle
integer                                 :: iter     !nombre d'itération
integer                                 :: iter_max !nombre d'itération max
real(8)                                 :: eps      !ce cher epsilon
character (len=20)                      :: nomfich  !nom du fichier

!affichage menu
print *,"------------------- tp3 : la methode QR -------------------"
print *,"Calcule le spectre d'une matrice donnée selon la methode QR"
print *,"-----------------------------------------------------------"


! Lecture du nom du fichier de donnees
print *,' Entrez le nom du fichier de donnees : '
read *,nomfich

!initialisation
open(1,file=nomfich,status="old",action="read")
read(1,*) n
allocate(A(n,n),spA(n))

DO i=1,n
    read(1,*) A(i,1:n)
END DO
close(1)

print *,"Veuillez saisir la valeur d'epsilon : "
read *,eps

print *,"Veuillez saisir le nombre d'iterations maximal : "
read *,iter_max

call afficheMat(A,n)
call QR(A,eps,iter_max,iter)

print *,"Nombre d'iterations : ",iter

!call afficheMat(A)
DO i=1,n
    spA(i) = A(i,i)
END DO

print *,"Le spectre de A : "
call afficheVect(spA,n)

print *,"-------------------------------------------------------"

END PROGRAM val_p
!fin du TP3\end{verbatim}
\end{small}

\newpage
\subsection{Module d'affichage : affichage.f90}
Le programme d'affichage provient du TP1, j'ai crée un module que je peux réutiliser
dans chaque programme afin d'afficher un vecteur ou une matrice. Ce module a déjà été
utilisé dans le TP2 sous le même nom.

\subsection{Module : algo.f90}
Ce module effectue l'algorithme de calcul des valeurs propres selon la methode QR décrite 
dans la partie \ref{algo}. On alloue dans ce module les matrices Q($n*n$), H($n*n$) et le 
vecteur v($n$) nécessaires à l'execution de l'algorithme. À la fin, on retourne la matrice A.

\begin{small}
\begin{verbatim}
MODULE algo
CONTAINS

SUBROUTINE QR(A,eps,iter_max,iter)
IMPLICIT NONE

real(8),dimension(:,:),intent(inout)    :: A        !matrice A
real(8),dimension(:),allocatable        :: v        !vecteur de calcul de Householder
real(8),dimension(:,:),allocatable      :: Q        !matrice de factorisation QR
real(8),dimension(:,:),allocatable      :: H        !matrice de Householder
real(8)                                 :: s        !calcul de la norme de A
real(8)                                 :: a_max    !composante maximale (non diag) de A
real(8),intent(in)                      :: eps      !epsilon

integer, intent(in)                     :: iter_max !iteration maximale
integer, intent(inout)                  :: iter     !compteur d'iterations
integer                                 :: n        !rang de A
integer                                 :: test     !valeur de test (pour l'iteration)
integer                                 :: i,j,k    !variables de boucle
integer                                 :: kp,km    !valeur de k+1 et k-1

!initialisation
n = size(A,dim = 1)
iter = 1
test = 0

!allocations
allocate(Q(n,n),H(n,n),v(n))

DO WHILE ((test == 0) .and. (iter < iter_max))

    iter = iter +1
    !initialisation de Q
    Q = 0
    DO i=1,n
        Q(i,i) = 1.d0
    END DO

    DO k=1,n-1

        !calcul de s
        s = 0.d0
        DO j=k,n
            s = s + A(j,k)**2
        END DO
        s = sqrt(s)

        !calcul de v
        kp = k+1
        km = k-1
        v(1:km) = 0
        v(k) = A(k,k)+A(k,k)/abs(A(k,k))*s
        v(kp:n) = A(kp:n,k)

        !calcul de r (-> s) et w (-> v)
        s = sqrt(2*s*(s+abs(A(k,k))))
        v = v/s

        !calcul de H
        DO i=1,n
            DO j=1,n
                IF (i == j) THEN
                    H(i,j) = 1-2*v(i)*v(j)
                ELSE
                    H(i,j) = -2*v(i)*v(j)
                END IF
            END DO
        END DO

        !calcul de Q et A
        Q = matmul(Q,H)
        A = matmul(H,A)

    END DO

    !mise à jour de A
    A = matmul(A,Q)

    !test sur les elements non diagonaux
    a_max = 0.d0
    DO i=1,n
        DO j=1,n
            IF (a_max < A(i,j)) THEN
                a_max = A(i,j)
            END IF
        END DO
    END DO

    IF (a_max < eps) THEN
        test = 1
    END IF

END DO

END SUBROUTINE QR

END MODULE algo
\end{verbatim}
\end{small}

\newpage
\section{Jeux d'essais}\label{essai}
On utilisera pour les essais une valeur de $\epsilon = 1.d-12$ et un nombre
d'itérations maximales égal à $100$.
\subsubsection{matrice41.txt}
$A=\left[\begin{array}{rrrr}
4 & 1 & 0 \\
1 & 4 & 1 \\
0 & 1 & 4
\end{array}\right]$

\begin{small}
\begin{verbatim}
 ------------------- tp3 : la methode QR -------------------
 Calcule le spectre d'une matrice donnée selon la methode QR
 -----------------------------------------------------------
  Entrez le nom du fichier de donnees : 
matrice41.txt
 Veuillez saisir la valeur d'epsilon : 
1.d-12
 Veuillez saisir le nombre d'iterations maximal : 
100
 La matrice A :
   4.0000000000000000        1.0000000000000000        0.0000000000000000
   1.0000000000000000        4.0000000000000000        1.0000000000000000
   0.0000000000000000        1.0000000000000000        4.0000000000000000
 Nombre d'iterations :          100
 Le spectre de A : 
   5.4142135623731233     
   3.9999999999999987     
   2.5857864376269060     
 -------------------------------------------------------
\end{verbatim}
\end{small}

\subsubsection{matrice42.txt}
$A=\left[\begin{array}{rrrr}
1 & 1 & 2 \\
1 & 2 & 1 \\
0 & 1 & 3
\end{array}\right]$

\begin{small}
\begin{verbatim}
 Nombre d'iterations :          100
 Le spectre de A : 
   4.0000000000000018     
   1.0101859644672808     
   0.98981403553271896 
\end{verbatim}
\end{small}

\subsubsection{matrice9.txt}
$A=\left[\begin{array}{rrrrrrrrr}
 4 & -1 &  0 & -1 &  0 &  0 &  0 &  0 &  0 \\
-1 &  4 & -1 &  0 & -1 &  0 &  0 &  0 &  0 \\
 0 & -1 &  4 &  0 &  0 & -1 &  0 &  0 &  0 \\
-1 &  0 &  0 &  4 & -1 &  0 & -1 &  0 &  0 \\
 0 & -1 &  0 & -1 &  4 & -1 &  0 & -1 &  0 \\
 0 &  0 & -1 &  0 & -1 &  4 &  0 &  0 & -1 \\
 0 &  0 &  0 & -1 &  0 &  0 &  4 & -1 &  0 \\
 0 &  0 &  0 &  0 & -1 &  0 & -1 &  4 & -1 \\
 0 &  0 &  0 &  0 &  0 & -1 &  0 & -1 &  4
\end{array}\right]$

\begin{small}
\begin{verbatim}
 Nombre d'iterations :          100
 Le spectre de A : 
   6.8284271247461996     
   5.4142135623731260     
   4.0001260626022601     
   5.4140874997708579     
   3.9999999999999902     
   3.9999998434059796     
   2.5857865942209264     
   2.5857864376269153     
   1.1715728752538108 
\end{verbatim}
\end{small}

\newpage
\section{Conclusion}\label{conclusion}
Ce TP permet le calcul des valeurs propres pour une matrice quelconque. Grâce à cette methode
on obtient cependant des valeurs approchées qui dépendent fortement du nombre d'itérations maximal
et de la précision souhaitée (ici on a pris $\epsilon{} = 1.d-12$ pour $100$ itérations.

Dans l'écriture du TP en fortran j'ai préféré ne pas utiliser trop de variables de stockage ($r$ et $w$) en
réecrivant les variables $v$ et $s$ au cours de l'alogorithme. Ceci permet un gain de place en mémoire (faible ??)
au détriment de la lisibilité du programme. Le fortran se révèle (encore une fois) très utile pour transcrire
ces algorithmes d'analyse numérique.

Enfin, comme pour le TP précédent j'ai rencontré des problèmes pour l'affichage des matrices dans le terminal.
J'ai résolu le problème en écrivant les matrices directement en \LaTeX{}.
\end{document}
